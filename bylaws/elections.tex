\chapter{Elections}\label{sec:elections}

\section{Moderator}\label{sec:moderator}
Elections are run by a moderator selected by a majority vote of the Coordinators, and may be any member not currently a coordinator and not running for a position in the election.

\section{Eligibility}\label{sec:eligibility}
All members are eligible to vote and run for open coordinator positions.

\section{Nominations}\label{sec:nominations}
Any member may nominate any other member or themselves for a coordinator position, nominations must be accepted by the nominee for them to become a candidate. Candidates must notify the Moderator of their intent to run. Each member may be a candidate for at most one coordinator position in an election.

\section{Platforms}\label{sec:platforms}
Candidates must submit their platforms to the Moderator in advance of the election for distribution in a message to all members.

\section{Voting}\label{sec:voting}
Voting takes place over a week and allows for online voting through a private system managed by the Moderator. Candidates may win through a plurality of the Membership voting, or a majority vote of the Coordinators in the event of a tie. At the end of the one week period the winners are considered ``Incoming Coordinators'' and elections are closed.

\section{Confidentiality}\label{sec:confidentiality}
The Moderator must only reveal how many members voted, which candidates won or tied which positions, and whether any amendments have passed. No other information (including which members voted which ways) should be released.

\section{Term Commencement}\label{sec:term_commencement}
After elections conclude a one month transition period starts in which the current Coordinators train the Incoming Coordinators. At the conclusion of this period the Incoming Coordinators assume their positions as Coordinators.
